\documentclass[fontsize=11pt]{article}  
\usepackage{amsmath}  
\usepackage{hanging}  
\usepackage[utf8]{inputenc}  
\usepackage[margin=0.75in]{geometry}  
  
\title{CSC110 Written Report: Observing the impact of the Covid-19 pandemic on travel in the United Kingdom between 2020 and 2021}  
\author{Jacob DeSousa, Marco Marchesano, Siddharth Arya}  
\date{Tuesday, December 14, 2021}  
  
\begin{document}  
\maketitle  
  
\section*{Problem Description and Research Question}  
  
In the past year the world as a whole has been affected by the Covid-19 pandemic. Around the world industries big and small have struggled, with revenue fluctuating just as much as confirmed cases. Small businesses were affected the most by the pandemic, however there is one large industry that has really struggled—transportation. With this in mind, our group will be looking at the impact Covid-19 has had on the travel industry between March 2020 and August 2021.
\\ \\
This time frame will allow our group to show how the industry was affected right at the start of the pandemic, and how it changed over the course of a year and a half as restrictions have slowly lifted. In the United Kingdom when the pandemic became the focus of the world in April 2020, public transit use declined 95\%. Although it recovered slowly as the world began to open up again, the significant drop in usage shows there is a correlation between how relevant Covid is and the use of different transportation methods
\\ \\
At the beginning of the pandemic travel virtually came to a halt, with very few people arriving and leaving their home country. Local transportation was also greatly affected as there was a fear of being in public spaces. Our group is going to attempt to find a correlation specifically between the confirmed Covid cases and the amount of transportation used in the United Kingdom. We will also observe how specific types of transportation's were affected and how there was an increase in private methods of transportation. Furthermore, not until recently have restrictions on travel been lifted, so it will be interesting to observe how the number of cases affected how much transportation was used. We want to look at the transportation sector because transportation is something that is heavily relied on in a non-Covid world. With the addition of the pandemic, it will be interesting to see how such a vital industry was affected.
\\ \\
With the materials and data sets we found we will attempt to answer the question: \textbf{Did confirmed Covid-19 cases have a relation to the amount transportation was used, if so how much was it affected? Also which methods of transportation's were affected the most? }


\section*{Dataset Description}

\begin{enumerate}
    \item Data Set 1 (Domestic transport use by mode: Great Britain, since 1 March 2020).
Usage of different modes of transportation (tracked through percentage - as compared to a set baseline), by date throughout the Covid-19 pandemic
    \begin{enumerate}
        \item From: gov.uk,  a repository of data collected by different government entities, all publish under the open government license (free to read and use)
        \item We will be modifying the file so that it only contains the time range from March 1, 2020 to October 25, 2021
        \item The format of the file is CSV and we will be using all of the available columns as follows
        \begin{enumerate}
            \item Date
            \item Cars
            \item Light Commercial Vehicle
            \item Heavy Good vehicle
            \item All Motor vehicles
            \item National Rail
            \item Transport for London Tube
            \item  Transport for London Bus
            \item Bus (excluding London)
            \item Cycling
        \end{enumerate}
    \end{enumerate}
    \item Data Set 2(Coronavirus (COVID-19) in the UK): New cases by date, and cumulative cases at the given date throughout the Covid-19 pandemic.
    \begin{enumerate}
        \item From: gov.uk,  a repository of data collected by different government entities, all publish under the open government license (free to read and use)
        \item We will be modifying the file so that it only contains the time range from March 1, 2020 to October 25, 2021
        \item The format of the file is CSV and we will be using the columns as follows
        \begin{enumerate}
            \item date
            \item newCasesBySpecimenDate
            \item cumCasesBySpecimenDate
        \end{enumerate}

    \end{enumerate}
\end{enumerate}

\section*{Computational Plan}

\begin{enumerate}
    \item We took the multiple csv files from our sources and extracted the data into PyCharm
    \item We wrote classes for data entries from our data sets. Specifically we will have a class for a piece of transportation data (ex Cars, Tube, Bus etc.), and a class for a piece of case data. Each class includes an attribute for each field in the original tables.
    \item Loaded data into Python using a load data function. The function returns a list of class instances named CaseData as well as TransportationData, each including one row of data from our data sets. This function was used on both data sets
    \item We created helper functions named average\_case\_data as well as average\_transport\_data that calculate the averages of the respective data in order to be displayed on the graph This function will return data averages over a given interval. For example, if you want to view data by month, the function returns an average of data from each month instead of the daily data that our data sets contain.
    \item We used plotly to create a user interface, allowing the data to be viewed in many different ways, each of the mode of transportation can be selected to be shown on the graph to help the user compare the data.
    \item When our program is run it will prompt the user to head to a web address where they can view a line graph of contaning the average Covid cases found over the given time interval, as well as the different forms of transportation used over the same interval. By displaying the data in this way it helps to answer or research question with the aid of a visual
    \item We also implemented a changeable scale. For example having time broken down by day, week, and month. It allows the user to scroll through a slider and select the data presented in the way they want (averaged over x number of days). This way we can account for outliers in the data, for instance if there was a lot more train travel than usual on a holiday, it would not affect the graph as much with a wider scale. This proves also to be useful for narrowing down on a particular range of data, say one particular wave of COVID cases.
    \item Finally, at the bottom of the page, there is another interactive section for the user, it allows the user to input a date (in the specified format) and returns the data for each variable on that given day.
\end{enumerate}

\section*{Instructions for Obtaining the data sets and running our program}
\begin{enumerate}
    \item Please go to the requirements.txt file to download all of the required libraries in order to run our program, if there is a problem installing a library please manually install all of the ones below
    \begin{enumerate}
        \item dash
        \item pandas
        \item requests
        \item plotly
    \end{enumerate}
    \item Next, to download all of the required data sets you will be required to run the main.py file and all of the files should be downloaded in the directory folder, you will not have to move the files, the file should put them in the correct directory
    \begin{enumerate}
        \item If there is a problem downloading the files please just run the get\_files.py file and they should all be downloaded
    \end{enumerate}
    \item When main.py is run you may have to go to your browser and enter the web address that is returned in the console, when this page loads there should be a graph displaying different data, also with the ability to modify how the data is being viewed. You will be able to select different methods of transportation and see how it compares to the confirmed Covid-19 cases in the United Kingdom. There will also be a slider at the bottom of the graph which will allow you to chose how smooth the graph is and provide a greater average of the data over the given time frame.
\end{enumerate}

\section*{Changes made since proposal}
Since the project proposal was submitted we removed the addition of international Canadian travellers from the data we are processing, we felt it would be much better if we focused on transportation in the UK and how it was affected by the Covid-19 pandemic. Having just one correlation to test for allowed us to dive more into different computations done through python. We also added a concrete example in the introduction of the report to show how transportation was affected at the start of the pandemic. Finally, modified the title of our project to better encompass the topic we are trying to cover. There are another few changes (small inclusions and exclusion) about what we are going to do with the data that should be aparent from the computational plan.


\section*{Discussion}

The research question are group attempted to answer was \textbf{Did confirmed Covid-19 cases have a relation to the amount transportation was used, if so how much was it affected? Also which methods of transportation's were affected the most?} \\ \\

By analysing different data sets from the government of the United Kingdom, our group can conclude there is a co-relation between the number of confirmed Covid-19 cases in the United Kingdom and the use of different transportation systems. Throughout the years our program examined it is clear to see that as the number of confirmed cases rose, the use of many transportation systems such as cars, the tube, busses etc., became less and less. We can see this trend was reversed as the number of cases goes down i.e, there was a negative correlation between the number of cases and the use of any mode of transportation. It is interesting to note however one of the findings we uncovered was if we look at the use of cars through March 2020, to October 2021, we can see that it does dip down as cases increase, but as cases decrease the percentage of cars almost returns to the amount it was at before. However, if we use a similar idea to compare the national rail or national buses, we can see that there usage never quite reached what it originally was before March 2020. We can attribute this discrepancy to the ways in which the pandemic affected the different modes of transportation: whilst cars/cycling is impacted by peoples reluctance to go out during the pandemic, national railways/buses may be changing mainly as a result of government regulation, therefore they are not perfectly proportionate. Through this analysis we can conclude that the usage of public transit continued to decline through the pandemic. \\ \\

The limitations with the datasets we found were the range in dates that were available, we would have liked to analyze some of the transportation usage before the beginning of the pandemic, but the data was not available from the government of the United Kingdom website. It would have been interesting to see if these numbers fluctuated as much as they did during the pandemic. Additionally, the data sets had some missing data which we had to account for whilst doing our computation. The process of using new modules also came with their own challenges, however those were to be expected when learning something unfamiliar. \\ \\

After looking at the results our group was able to create through our program, an area of further exploration would be seeing if these trends are like different countries around the world. We were able to conclude that there is a correlation between the number of confirmed Covid-19 cases and the modes of transportation, however, can we find this correlation in other countries outside the United Kingdom. More specifically comparing a developed and a developing country to see if these trends are the same. One country that would be very interesting to analyze would be the United States as they handled Covid very differently than the rest of the world, one could compare similar statistics with the ones from the United Kingdom and see how the forms of transportation are impacted. Finally taking this idea one step further one could further explore the use of air travel in relation to confirmed Covid 19 cases, and observe if the use of air travel is being used as much as it was before the pandemic \\ \\

In conclusion, our group was able to successfully take a research area in question, analyze available data and formulate a proper response to the question we had originally asked. The dash library had lots of available tools for our group to use to further analyse our research question and visualize an answer he area of interest we chose leaves lots of room for further exploration and would be very interesting to further analyze.


\section*{References}
\begin{hangparas}{.25in}{1}
“Mind the Gaps: Will We Go Back to Public Transport after Covid?” The Guardian, Guardian News and Media, 20 Mar. 2021, www.theguardian.com/business/2021/mar/20/mind-the-gaps-will-we-go-back-to-public-transport-after-covid. \\

\textit{Plotly python graphing library. } Plotly. (n.d.). Retrieved October 29, 2021, from https://plotly.com/python/. \\


Transport, D. for. (2021, October 27). \textit{Transport use during the coronavirus (COVID-19) pandemic.} GOV.UK. Retrieved October 29, 2021, from https://www.gov.uk/government/statistics/transport-use-during-the-coronavirus-covid-19-pandemic. \\


\textit{United Kingdom.} Coronavirus (COVID-19) in the UK. (n.d.). Retrieved October 29, 2021, \\ from https://coronavirus.data.gov.uk/.

% NOTE: LaTeX does have a built-in way of generating references automatically,
% but it's a bit tricky to use so we STRONGLY recommend writing your references
% manually, using a standard academic format like APA or MLA.
% (E.g., https://owl.purdue.edu/owl/research_and_citation/apa_style/apa_formatting_and_style_guide/general_format.html)

\end{hangparas}
\end{document}
